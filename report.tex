\documentclass[a4paper, titlepage]{article}

\usepackage{courier} % Required for the courier font
\usepackage[utf8]{inputenc}
\usepackage{graphicx}
\newcommand{\rr}{Ret\&Råd}

\begin{document}

\title{Individual Reflective Report for the Design Project}
\author{Sigurt Bladt Dinesen\\
sidi{@}itu.dk\\
Group 3\\
Business Processes and Organisation
}

\maketitle

\section*{Interviewing subjects}
This reflection report offers a critical look at the interviews performed in the
course of our project; how they were performed, and concludes by
suggesting changes for improvement.

\subsection*{What we did}
The interviews we performed can be divided into two parts: Meetings and initial
interviews with \rr{} CEO, Casper Hauberg Grønnegaard, and later empirical
data gathering with office counselors across the organization.
While the size of our dataset may be insufficient for real analysis, we were
completely dependent on the benevolence of \rr{}, and are thankful for the
time they agreed to spend with us.

The "trick" to an interview is realising that it is an economy of knowledge,
albeit a strange one. In order to perform a successful interview one must, to
some extend, have control of all of the following:
\begin{enumerate}
	\item \label{we wish} What do we wish to learn
	\item \label{think wish} What does the interviewee think we wish to learn
	\item \label{they know} What does the interviewee know that we do not
	\item What do we know that the interviewee thinks we do not
	\item What do we know that interviewee should not
	\item What does the interviewee know - and think we should not.
\end{enumerate}
The importance of these points may seem obvious, but the full meaning was not when
we began our work with \rr{}.

Point \ref{we wish} and \ref{they know} are tied together; it is vital that the
chosen subject possesses the knowledge we seek. This was our only basis for the
initial interviews with Casper. With the \rr{} website as our only source of
information, we \textit{knew} nothing, and he knew everything. Our goal was to
obtain any knowledge regarding the organization. Our chosen interview strategy
proved a good fit for this situation: The meetings with Casper alternated
between unstructured and semi-structured interviews, always with one
interviewer and two or three note takers. This method was successful, not least
because of Casper's openness, but also because it allowed one focused
interviewer to guide the conversation though a broad spectrum of topics.

For interviewing paralegals and lawyers, we used a similar method, only more
structured, with actual interview guides. Initially it seemed to be effective.
We had a better idea of what knowledge we sought to expand, and used it to seek
subjects that possessed that knowledge. After a while, the disregard of point
\ref{think wish} became evident though. The interviewee, Pernille, kept
answering questions I (the interviewer for the day) had not meant to ask, and it
must have seemed to her like I kept repeating questions. At one point,
the present group members agree, I came close to putting words in the
mouth of the interviewee, as an ill fated attempt at making the question
clear\footnote{The question was about the value of searchable cases versus
verdicts, and has since become irrelevant to the business case.}.

During the course of the project we adjusted the questions. Adjustments were due
to a wish for new data, direct changes in our project charter, or problems
experienced in previous interviews.

\subsection*{What we did wrong/could have done}
The interview strategy we have used is in agreement with recommendations made in
Bødker et al, chap. 9.4, but failed to support changes in our basis of, and need
for, knowledge. As we moved on to the last couple interviews with Casper
interviews with the counsels, our interviews stayed very open and unstructured.
While there may be nothing wrong with that, it made it difficult for the
inexperienced interviewer to stay on track, and to choose the right deviations
from the interview guides. A more extensive preparation for the individual
interviews may have lessened the problem, but an increased focus on semantics of
individual questions seem inherent to structured (or at least more structured)
interviews.

I can not \textit{reccommend} a different course of action, but for another
project I would like to try readjusting the interview strategy according to
the progress of the project.

\end{document}
